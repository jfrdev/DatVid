\documentclass{beamer}
\usepackage[danish]{babel}
\usepackage{ucs}
\usepackage[utf8x]{inputenc}
\usepackage[T1]{fontenc}
\usepackage[danish]{babel}
\usepackage{libertine}

\begin{document}
\begin{frame}
\frametitle{Den matematiske tilgang}
\begin{itemize}
    \item Historisk faktum: Datalogi har sine rødder i matematikken
    \item Beregnelighed -- hvad kan beregnes, hvad kan ikke
    \item Beviser -- Hypotetiko-deduktive metode
\end{itemize}
\end{frame}

\begin{frame}
\frametitle{Beviser}
\begin{itemize}
    \item Matematik: ``Simple'' beviser
    \item Programmer: Mulighed for meget store korrekthedsbeviser
    \item Programmer kan være så komplekse at de praktisk talt ikke kan bevises
    matematisk
    \item Programmer kan ``bevises'' empirisk
\end{itemize}
\end{frame}

\begin{frame}
\frametitle{Theory vs. practice}
\begin{itemize}
    \item Matematiske korrekthedsbeviser er irrelevant hvis det ikke kan
    anvendes i praksis
    \item Matematikken er begrænset af den fysiske verden
    \item Modargument for, at det er en ren matematisk videnskab
\end{itemize}
\end{frame}

\begin{frame}
\frametitle{Turingmaskinen vs. Lamdakalkulen}
\begin{itemize}
    \item Samme matematiske udtrykskraft
    \item Egnet i forskellige scenarier (eks: samtidighed)
    \item Illusterer forskellen på matematisk og naturvidenskabelig metode
\end{itemize}
\end{frame}
\end{document}
