\documentclass[10pt,a4paper]{article}

\usepackage[danish]{babel}
\usepackage{ucs}
\usepackage[utf8x]{inputenc}
\usepackage[T1]{fontenc}
\usepackage[danish]{babel}
%\usepackage{courier, fourier}
\usepackage{libertine}
%\usepackage[pdftex]{graphicx}
%\usepackage{float}
%\usepackage[table,x11names]{xcolor}
%\usepackage{mdwlist}
\usepackage[parfill]{parskip}


%\linespread{1.2}

%\setcounter{secnumdepth}{2}
%\setcounter{tocdepth}{2}

\widowpenalty=1000
\clubpenalty=1000

\setlength{\parskip}{3ex plus 2ex minus 2ex}

\newcommand{\citat}[2]{\vspace{0.1cm}\newline\textit{``#1''}\hspace{0.1cm}\footnote{#2}\vspace{0.1cm}\newline}

\begin{document}
\section{Indledning}
Rapporten forsøger at beskrive og sammenligne de mest væsentlige punkter fra de
tre givne artikler. Vi har derfor valgt at fremhæve dels de forskellige
synspunkter på faget datalogi/computer science og dels de forskelle og ligheder
der er mellem fagene af henholdsvis Peter Naurs fortolkning, datalogi, og den
amerikanske fortolkning, computer science.  Generelt om sammenligningerne skal
man huske på, at artiklerne er fra henholdsvis slutningen af 60'erne og 2008,
altså fra meget forskellige tider, i hvert fald ift. IT, og derfor er også deres
grundlag forskelligt. Peter Naur skulle lave et udkast til hans idé om faget
datalogi (som han selv insisterede på skulle hede sådan), altså et helt nyt fag,
mens curriculum 2008 er en opdatering/forbedring af computer science og bygger
på mange årtiers erfaring.  Nogle af de største forskelle på de to uddannelser
er den generelle opfattelse af faget og dets sammenhæng, samt hvordan faget skal
undervises.

\section{Datalogi og Computer Science}
Ved første gennemlæsning af Peter Naurs Rosenkjær-foredrag om datalogifaget
\footnote{Datalogi - læren om data}
skiller én formulering sig ud som opsigtsvækkende, navnlig:
\citat{Det afgørende er indholdet af undervisningen.  Hovedtemaerne må være
    data, datarepræsentationer, og dataprocesser. [...] Datamaterne bør også
    omtales, men ikke som det centrale i faget, snarere som en afsluttende
    orientering.}{Peter Naur: \emph{Datalogi: læren om data}, Datalogisk 
    Institut 1967, s. 15}
På en datalogistuderende i dette årtusind kan det
måske forekomme kontraintuitivt overhovedet at omtale datalogi som disciplin
uden maskinen som primært fokus. Med den overældende hastighed der har præget
den teknologiske udvikling siden fagets grundlæggelse er datamatens ydeevne
og kompleksitet øget så kraftigt at det alene forekommer som rigeligt objekt
for en akademisk disciplin. Et centralt punkt i Peter Naurs tekst her er da også
opfattelsen af datamatens rolle i denne forbindelse. Som det pointeres i
tekstens indledning ses datamater som:
\citat{[...] datamanipulatorer, og ikke i første række som regnemaskiner.}
{Ibid. s. 10}. 
 
\subsection{Databegrebet}
Et nærliggende spørgsmål bliver i denne kontekst hvad præcies der menes med
ordet ``data''. Naur fremhæver selv den internationale defintion af ordet som:
\citat{[...]enhver repræsentation af fakta eller idéer på en formaliseret måde, som
kan kommunikeres eller manipuleres ved en eller anden proces.}{Ibid. s. 10}.
Naur fremhæver to centrale aspekter ved denne definition, først og fremmest
dens notoriske vaghed: netop det faktum at ``data'' kan dække over en
mangfoldighed af fænomener. En anden central egenskab ved databegrebet er dets 
intime forbindelse til processer: \citat{En ting eller begivenhed er ikke i sig 
selv data, men bliver det først når den indgår i en proces hvori dens repræsentation 
af fakta og idéer er det afgørende}{Ibid., s. 11}. Studiet af data består således i
høj grad af undersøgelser af hv


\citat{Sagen er at den lethed 
hvormed vi kan gennemføre en given dataproces i høj grad afhænger af præcis hvilke
datavi har valgt til at repræsentere virkeligheden, eller som jeg kort vil sige,
af \emph{datarepræsentationen}.}{Ibid., s. 12} 


\subsection{Datalogi i forhold til Computer Science}
Datalogi er således studiet af data og dets organisation - dette er som sådan 
uafhængigt af den maskinelle behandling dataet udsættes for. Datamaten opnår sin 
eksistensberettigelse som datalogisk studieobjekt grundets dens evne til at 
``frigøre'' dataet som fænomen. Naur udtrykker det \citat{Fremkomsten af datamaterne
betyder at langt mere komplicerede dataprocesser meget effektivt kan udføres
uden at mennesker behøver at deltage. Vejen er derved åbnet for at arbejdet
 med datamodeller bliver i høj grad automatiseret og effektiviseret.}{Ibid., s. 13}

\citat{[...] databegrebet
    beskæftiger sig med en måde som mennesker forholder sig på til visse
    fænomener.}{Ibid., s. 11} Lav sektion om def. af data: Beskrivelse af hvordan
kunst, modeller og data er forbundet.  \citat{Sagen er at den lethed hvormed vi
    kan gennemføre en given dataproces i høj grad afhænger af præcis hvilke data
    vi har valgt til at repræsentere virkeligheden, eller som jeg kort vil sige,
    af \emph{datarepræsentationen}.}{Ibid., s. 12} 
\citat{befrier [...] os for fordomme om at arbejdet med en bestemt given
    problemstilling er knyttet til en bestemt datarepræsentation}{Ibid., s. 14}

\subsection{Applikationsorientering}
En lighed mellem den måde datalogi fremlægges på i de tidlige tekster, i forhold
til hvordan det fremlægges af ACM, er at begge lægger vægt på den
applikationsorienterede tilgang til faget.

Omhandlede undervisningen på DIKU i 1970 (da DIKU blev oprettet), beskrives:
\citat{Through their project work, which occupies half of their study time, the
    students get the opportunity to use theoretically learned material to solve
    actual problems.}{Datalogy -- The Copenhagen Tradition of Computer
    Science, s. 454}

Envidere findes følgende citat fra Peter Naur: \citat{One will always be faced
    with the difficulty of deciding wheter what one is doing is scientifically
    defensible, if it is valuable enough. This kind of nagging doubt is unknown
    in the pure subjects. There, researchers adopt their own basis of evaluation
    independently of demands made by the complicated and unclean reality.
    However, at DIKU we have hitherto been able to maintain our
    applicationsoriented line, and we have attracted many students who have been
    able to use what we try to teach them.}{Citat fra Peter Naur, Ibid., s. 469}

Disse citater, i særdeleshed det første, viser hvordan man i den københavnske
tradition lagde meget vægt på projektarbejde, og derved havde en
applikationsorienteret tilgang til indlæringen af faget. Det var altså i Naurs
tilgang en vigtig del af datalogien, at denne i særdeleshed var et værktøj til
at løse problemer med, og at man derfor som en fuldstændig central del måtte
benytte sig af projektarbejde.

At dette også til en hvis grænse gør sig gældende for ACM kan ses i følgende
citat, som beskriver en af de vigtige færdigheder som studerende skal mestre
efter endt uddannelse: \citat{\emph{Significant project experience}. To ensure
    that graduates can successfully apply the knowledge all students in computer
    science programs must be involved in at least one substantial software
    project (usually positioned late in a program of study) demonstrates the
    practical application of principles learned in different courses and forces
    students to integrate material learned at different stages of the
    curriculum. Student need to appreciate the need for domain knowledge for
    certain applications, and this may necessiate study whitin that
    domain.}{Computer Science Curriculum 2008, kap. 4.2}

Her lægges altså også vægt på at studerende skal have projekterfaring, og at de
som minimum skal indgå i et større projekt (hvilket dog er væsentligt mindre end
den anden tilgang). Dog er dette rimelig uforpligtende, og der skrives ikke
yderligere om projektarbejde i teksten.

\section{Fag}
\subsection{Værktøjer og sprog}
Om ACM curriculum 68\footnote{Datalogy -- The Copenhagen Tradition of Computer
    Science, s. 459} beskrives, hvorledes det undgår at have en overordnet
vinkel til programmering og teknikker i følgende citat fra Peter Naur:
\citat{The deepest difference is that the ACM curriculum (of 68, red) [...]
    apparently making sure to mention all the current techniques, languages and
    practices, however briefly, while the datalogy course strives to emphasize
    the underlying ideas and principles, while omitting many particular
    instances of the various notions. [...] By keeping these matters out of the
    course of datalogy, this cna concentrate on the basic matters, common to all
    environments.}{Citat fra Peter Naur, Datalogy - The Copenhagen tradition of
    computer science, s. 459}

Naur beskriver også faget ``Computers and programming languages'', i.e. flere
programmeringssprog, i hans forslag til emner til datalogifaget. 

Dette problem har ifølge ACM curriculum 2008 eksisteret indtil nu, hvilket
understreges ved at der fra industrien bliver kommenteret på problemet:
\citat{An emphasis was placed on the problems of students having been
    indoctrinated in particular tools or processes that they have to
    unlearn.}{Computer Science Curriculum 2008, kap. 1.2}

Det er altså klart, at som datalog skal man have en grundlæggende og abstrakt
viden om programmering og processer, så man hurtigt kan arbejde med effektivt i
industrien, hvor mange forskellige sprog og processer bruges. Det er også noget,
curriculum 2008 har fokus på, og anbefaler at studerende får kendskab til mindst
et programmeringssprog.

\subsection{Debugging og metoder}
Peter Naur havde også fokus på problemet i at teste og bevise korrekthed af
algoritmer og kode\footnote{Datalogy - The Copenhagen tradition
    of computer science, s. 464}. Dette bliver også fremstillet som et vigtigt punkt i
curriculum 2008, hvor der i følge industrien bør være større fokus på
kvalitetssikring. 

\subsection{Menneske/maskine-interaktion}
Allerede om curriculum 68 kommenterede Peter Nauer på, at der ikke optrådte
nogen behandling af faget menneske-maskin interaktion. Dette optræder dog i
curriculum 2008 i både ``Knowledge areas'', som Human-Computer Interaction, og
som en færdighed en uddannet studerende skal have. Det skal dog overvejes,
hvorvidt Peter havde samme idé med faget ``Man/machine communication'', idet der
på tidspunktet ikke var samme teknologiske muligheder (eksempelvis bestod GUI'en
af en terminal, hvor der i dag er mange flere muligheder). Det er dog essentielt
det samme fokus i dag, nemlig kommunikationen mellem mennesket og computeren, og
med mere fokus på interaktionen, der fylder meget mere.

\subsection{Projekter, store datasystemer og oversætter}
Industrien anbefaler kraftigt, at de studerende skal have
projektkurser/projektforløb, idet det giver dem erfaring i den praksis, der
eksistere i dagens industri, nemlig store projekter. Dette var noget Peter Naur
fra starten af argumenterede for, idet han erfarede, at en praktisk del var
essentiel for at udvikle forståelse (intuition) for generel,
programmeringsorienteret problemløsning, herunder de processer der ligger til
grund for projekter. Altså hvordan det foregår i industrien. Derfor sørgede han
også for, at halvdelen af datalogiuddannelsen bestod af projektarbejde. Dette
ligger også til grund for en af de største forskelle på de to fortolkninger,
datalogi og computer science, idet computer science hovedsageligt er
\citat{[...] The more academic, ``pure'' computer science oriented study of
    programming [...]}{Datalogy - The Copenhagen tradition of computer science,
    s. 469} mens datalogi er \citat{[...] The world of practical
    programming.}{Datalogy - The Copenhagen tradition of computer science, s.
    469}

Derudover anbefalede industrien også oversætter som kursus, idet det var et
lille, men realistisk softwareudviklingprojekt, på trods af at firmaerne, der
anbefalede kurset, ikke selv producerer oversættere. Dette var igen noget Peter
Naur foreslog som kursus, netop med henblik på store datasystemer, i.e. hvad den
virkelige verden arbejder med.

Det skal dog nævnes, at på trods af industriens anbefalinger, er hverken
oversætter eller projekter/projektkurser blevet en del af pensummet i curriculum
2008, men kun anbefalinger fra ACM's side til hvorledes kompetencerne kan
udvikles igennem. Dette er nok også påvirket af, at grundlaget til computer
science ikke er praktisk orienteret, som ved datalogi.

\subsection{Netværk og sikkerhed}
Curriculum 2008 har stor opmærksomhed på den stigende udvikling af
sikkerhedskrav indenfor både netværk og applikationer. Dette var ikke noget
Peter Naur havde stort fokus på, i hvert fald ikke i forbindelse med ulovlig
indtrængen (hacking). Peter Naur havde derimod kun fokus på sikkerhed af
udførelsen af applikationen, altså hvorvidt programmet var korrekt. Dette hænger
naturligvis sammen med tidsforskellen, hvor udbredelsen af store netværk som
internettet først rigtigt kom på banen i 90'erne med browseren, og med dette kom
et stigende sikkerhedsproblem.

\section{Konklusion}
Konkluderende kan siges at den store forskel mellem de to tekste ligger i at
Naur arbejder ud fra begrebet datalogi, mens ACM arbejder ud fra begrebet
computer science. 

Generelt er mange af de emner/problemstillinger, som Peter Naur tidligt
kommenterede på, bl.a. i forbindelse med curriculum 68, blevet introduceret
i curriculum 2008, som følge af industriens kritik af de uddannede studerene
inden for faget computer science i USA.
Derudover er der, som følge af den teknologiske udvikling, opstået nye
problemstillinger som netværkssikkerhed og sikkerhed mod ulovlig brug af
applikationer, der er af stigende interesse fra industrien. Dette er
selvfølgelig ikke noget, man kunne forudse i '60'erne, men er stadig en
væsentlig forskel på fagene.
\end{document}
