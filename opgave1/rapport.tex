\documentclass[10pt,a4paper]{article}

\usepackage[danish]{babel}
\usepackage{ucs}
\usepackage[utf8x]{inputenc}
\usepackage[T1]{fontenc}
\usepackage[danish]{babel}
\usepackage{courier, fourier}
\usepackage[pdftex]{graphicx}
\usepackage{float}
%\usepackage[table,x11names]{xcolor}
\usepackage{mdwlist}
\usepackage[parfill]{parskip}

%\linespread{1.2}

%\setcounter{secnumdepth}{2}
%\setcounter{tocdepth}{2}

\widowpenalty=1000
\clubpenalty=1000

\setlength{\parskip}{3ex plus 2ex minus 2ex}

\begin{document}
\section{Indledning}

TODO: Afdæk, beskriv og diskutér væsentlige forskelle og ligheder mellem
“Computer Science Curriculum 2008” og Peter Naurs meget tidlige beskrivelser af
faget datalogi, som de findes beskrevet i


\section{Forskellen mellem ``Datalogi'' og ``Computer Science''}
Ved første gennemlæsning af Peter Naurs Rosenkjær-foredrag om 
datalogifaget\footnote{Peter Naur: \emph{``Datalogi - læren om data''}, Datalogisk
Institut 1967.} skiller én formulering sig ud som opsigtsvækkende, navnlig: 
``Det afgørende er indholdet af undervisningen. Hovedtemaerne må være data, 
datarepræsentationer, og dataprocesser. [...] Datamaterne bør også omtales, men
ikke som det centrale i faget, snarere som en afsluttende orientering.''
\footnote{Ibid, s. 15}. På en datalogistuderende i dette årtusind kan det 
måske forekomme kontraintuitivt overhovedet at omtale datalogi som disciplin
uden maskinen som primært fokus. Med den overældende hastighed der har præget
den teknologiske udvikling siden fagets grundlæggelse er datamatens ydeevne og 
kompleksitet øget så kraftigt at det alene forekommer som rigeligt objekt for 
en akademisk disciplin. Et centralt punkt i Peter Naurs tekst her er da også 
opfattelsen af datamatens rolle i denne forbindelse. Som det pointeres i tekstens 
indledning ses datamater som ``[...] datamanipulatorer, og ikke i første række
som regnemaskiner.''. 
\subsection{Databegrebet}
Et nærliggende spørgsmål bliver i denne kontekst hvad præcies der menes med
ordet 'data'. Naur fremhæver selv den internationale defintion af ordet som 
``enhver repræsentation af fakta eller idéer på en formaliseret måde, som kan 
kommunikeres eller manipuleres ved en eller anden proces.''. Naur fremhæver selv 
to centrale aspekter ved denne definition, først og fremmest dens notoriske vaghed: 
netop det faktum at 'data' kan dække over en mangfoldighed af fænomener. En anden
central egenskab ved databegrebet er dets intime forbindelse med 

\subsection{Datalogi vs. CS}


``En ting eller begivenhed er ikke i sig selv data, men bliver det først når den
indgår i en proces hvori dens repræsentation af fakta og idéer er det afgørende''
(s. 11)

``[...] databegrebet beskæftiger sig med en måde som mennesker forholder 
sig på til visse fænomener.'' (s. 11)

Lan sektion om def. af data: Beskrivelse af hvordan kunst, modeller og data er
forbundet. 

Sagen er at den lethed hvormed vi kan gennemføre en given dataproces i høj grad 
afhænger af præcis hvilke data vi har valgt til at repræsentere virkeligheden, 
eller som jeg kort vil sige, af \emph{datarepræsentationen}. 
(s. 12)

Fremkomsten af datamaterne betyder at langt mere komplicerede dataprocesser 
meget effektivt kan udføres uden at mennesker behøver at deltage. Vejen er 
derved åbnet for at arbejdet med datamodeller bliver i høj grad automatiseret og 
effektiviseret. (s. 13)

``befrier [...] os for fordomme om at arbejdet med en bestemt given problemstilling
er knyttet til en bestemt datarepræsentation'' (s. 14)

``Det afgørende er indholdet af undervisningen. Hovedtemaerne må være data, 
datarepræsentationer, og dataprocesser. [...] Datamaterne bør også omtales, men
ikke som det centrale i faget, snarere som en afsluttende orientering.'' (s. 15). 


\subsection{Applikationsorientering}
En lighed mellem den måde datalogi fremlægges på i de tidlige tekster, i forhold til hvordan det fremlægges i pensummet fra 2008, er at begge lægger vægt på den applikationsorienterede tilgang til faget.

Omhandlede undervisningen på DIKU i 1970 (da DIKU blev oprettet), beskrives:
``Through their project work, which occupies half of their study time, the students get the opportunity to use theoretically learned material to solve actual problems.''\footnote{s. 454, Datalogy -- The Copenhagen Tradition of Computer Science.}

Envidere findes følgende citat fra Peter Naur:
``One will always be faced with the difficulty of deciding wheter what one is doing is scientifically defensible, if it is valuable enough. This kind of nagging doubt is unknown in the pure subjects. There, researchers adopt their own basis of evaluation independently of demands made by the complicated and unclean reality. However, at DIKU we have hitherto been able to maintain our applicationsoriented line, and we have attracted many students who have been able to use what we try to teach them.''\footnote{s. 469, Datalogy -- The Copenhagen Tradition of Computer Science.}

Disse citater, i særdeleshed det første, viser hvordan man i den københavnske tradition lagde meget vægt på projektarbejde, og derved havde en applikationsorienteret tilgang til indlæringen af faget. %TODO WRITE MOARH OK

At dette også gør sig gældende for pensummet fra 2008 kan ses i følgende citat:
``\emph{Significant project experience}. To ensure that graduates can successfully apply the knowledge all students in computer science programs must be involved in at least one substantial software project (usually positioned late in a program of study) demonstrates the practical application of principles learned in different courses and forces students to integrate material learned at different stages of the curriculum. Student need to appreciate the need for domain knowledge for certain applications, and this may necessiate study whitin that domain.''
%TODO skriv om dette citat

\section{Fag}
\subsection{Værktøjer og sprog}
Om ACM Curriculum 68 beskrives, hvorledes det undgår at have en overordnet vinkel til programmering og teknikker i følgende citat fra Peter Naur:
``The deepest difference is that the ACM Curriculum (of 68, red) [...] apparently making sure to mention all the current techniques, languages and practices, however briefly, while the datalogy course strives to emphasize the underlying ideas and principles, while omitting many particular instances of the various notions. [...] By keeping these matters out of the course of datalogy, this cna concentrate on the basic matters, common to all environments.'' (citat af Naur fra side 459 af Datalogy - the Copenhagen tradition of computer science)

Naur beskriver også faget ``Computers and programming languages'', i.e. flere programmeringssprog, i hans forslag til emner til datalogifaget. 

Dette problem har åbenbart eksisteret indtil nu, hvilket også understreges i rapporten om ACM curriculum fra 2008, hvor der fra industrien bliver kommenteret på problemet: 
``An emphasis was placed on the problems of students having been indoctrinated in particular tools or processes that they have to unlearn.'' (citat fra Curriculum 2008, kap. 1.2)

Det er altså klart, at som datalog skal man have en grundlæggende og abstrakt viden om programmering og processer, så man hurtigt kan arbejde med effektivt i industrien, hvor mange forskellige sprog og processer bruges. Det er også noget, Curriculum 2008 har fokus på, og anbefaler at studerende får kendskab til mindst et programmeringssprog.

\subsection{Debugging og metoder}
Peter Naur havde også fokus på problemet i at teste og bevise korrekthed af algoritmer og kode (se s. 464 fra af Datalogy - the Copenhagen tradition of computer science). Dette bliver også fremstillet som et vigtigt punkt i Curriculum 2008, hvor der i følge industrien bør være større fokus på kvalitetssikring. 

\subsection{Menneske/maskine-interaktion}
Allerede om Curriculum 68 kommenterede Peter Nauer på, at der ikke optrådte nogen behandling af faget menneske-maskin interaktion. Dette optræder dog i Curriculum 2008 i både ``Knowledge areas'', som Human-Computer Interaction, og som en færdighed en uddannet studerende skal have.

\subsection{Store datasystemer}


\subsection{Projekter og oversætter}


\subsection{Konklusion}
Generelt er mange af de emner/problemstillinger, som Peter Naur tidligt kommenterede på, blevet introduceret i Curriculum 2008, som følge af industriens kritik af de studerene inden for faget computer science i USA.


\section{Konklusion}

\end{document}