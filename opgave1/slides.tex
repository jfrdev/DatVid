\documentclass{beamer}
\usepackage[danish]{babel}
\usepackage{ucs}
\usepackage[utf8x]{inputenc}
\usepackage[T1]{fontenc}
\usepackage[danish]{babel}
\usepackage{courier, fourier}


\begin{document}
\begin{frame}
\frametitle{Data vs. maskine}
Peter Naur om undervisning i datalogi:

``Hovedtemaerne må være data, 
datarepræsentationer, og dataprocesser. [...] Datamaterne bør også omtales, men
\emph{ikke som det centrale i faget}, snarere som en afsluttende orientering.''
(egen kursivering)

Peter Naur:\emph{``Datalogi - læren om data''}, Datalogisk Institut 1967.
\end{frame}

\begin{frame}
\frametitle{Databegrebet}
En ting eller begivenhed er ikke i sig selv data, men bliver det først når den
indgår i en proces hvori dens repræsentation af fakta og idéer er det afgørende''

Ibid. s. 11
\end{frame}

\begin{frame}
\frametitle{Datamaten}
Peter Naur om datamatens primære rolle: 

``[...] datamanipulatorer, og ikke i første række som regnemaskiner.''

Ibid. s. 10 \newline

``Fremkomsten af datamaterne betyder at langt mere komplicerede dataprocesser 
meget effektivt kan udføres uden at mennesker behøver at deltage. Vejen er 
derved åbnet for at arbejdet med datamodeller bliver i høj grad automatiseret og 
effektiviseret.''

Ibid. s. 13 \newline

Datamaten ``befrier [...] os for fordomme om at arbejdet med en bestemt given
problemstilling er knyttet til en bestemt datarepræsentation''

Ibid. s. 14
\end{frame}
  
\begin{frame}
\frametitle{Fag}
\begin{itemize}
    \item{Værktøjer og sprog}
    \item{Kvalitetstesting af kode og algoritmer}
    \item{Projekter og oversætter}
    \item{Human/computer interaction}
    \item{Sikkerhed}
    \item{Net-orienteret}
\end{itemize}
\end{frame}
\end{document}