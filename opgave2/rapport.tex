\documentclass[10pt,a4paper]{article}

\usepackage[danish]{babel}
\usepackage{ucs}
\usepackage[utf8x]{inputenc}
\usepackage[T1]{fontenc}
\usepackage[danish]{babel}
\usepackage{libertine}
\usepackage[parfill]{parskip}
\usepackage{ragged2e}

\widowpenalty=1000
\clubpenalty=1000

\newcommand{\citat}[2]{\begin{justify}\textit{``#1''}\hspace{0.1cm}\footnote{#2}\end{justify}}

\title{Gruppeopgave 2}
\author{Henrik Bendt (191191)\\Jens Fredskov (240191)\\Naja Wulff Mottelson (300988)}
\date{\today}

\begin{document}
\maketitle

\section{Indledning}
Vi har opdelt beskrivelserne af de tre forskellige muligheder for
implementation af den fysiske computer, som har umiddelbart mest potentiale.
Den første er den traditionelle transistorbaserede computer, den anden er den
fysikorienterede kvantebaserede computer og den sidste er den
biologiorienterede molekulære DNA-computer. De adskiller sig alle på både deres
videnskabelige baggrund, fysiske form, potentialer, muligheder og
begrænsninger. Idet vi er nået en hvis form for hastighedsbegrænsning med den
transistorbaserede computer (ift. clock-hastighed og energieffektivitet mod
varmeafledning), og har mange problemer, der tager meget lang tid at udregne (op
til årtier), er der selvfølgelig muligheden for at se på alternative computere,
der måske effektivt kan løse nogle af de problemer, transistorcomputeren ikke
kan.

\section{Den transistorbaserede computer}
\subsection{Situationen før i tiden}
Hvis man først ser på Gordon E. Moores artikel \textit{Cramming more components
onto integrated circuits} fra 1965, er synet på fremtiden rimelig positivt. Om
forventningerne skrives bl.a.: \citat{The complexity for minimum component costs
has increased at a rate of roughly a factor of two per year (see graph on next
page). Certainly over the short term this rate can be expected to continue, if
not to increase. [...] there is no reason to believe it will not remain nearly
constant for at least 10 years. That means by 1975, the number of components per
integrated circuit for minimum cost will be 65,000. I believe that such a large
circuit can be built on a single wafer.}{\textit{Cramming more components onto
integrated circuits}, s. 2}

Her beskrives netop en forventning om, at man vil kunne bygge billigere kredsløb
med en højere densitet, år for år. Ser vi på historien har Moores forudsigelse h
oldt stik, frem til inden for de sidste 10 år ca. hvor man grundet
varmeudvikling (hvilket Moore nævner, men ikke mener vil være et problem) har
været nødt til at se på andre løsninger end blot at øge antallet af
transistorer.

\subsection{Situationen i dag}
For at se hvordan det ser ud for den transistorbaserede computer i dag kan man
efterfølgende se på artiklen \textit{Supercomputere -- Mange bække små...} af
Brian Vinter, fra 2010. Denne beskriver hvordan vi i dag har ramt en mur
for hastigheden af en enkelt processor.

Dette skyldes, at det at øge frekvensen yderligere resultere i så høje
temperaturer, at man med den tidligere udvikling, i 2007 skulle have en 16 GHz
processor med en temperatur på godt og vel 6000 grader celsius. Derfor er denne
udvikling stoppet, og i stedet må man kigge nye veje for at fortsætte med at
presse ydelse ud af den transistorbaserede computer.

Specfikt nævner Vinter at Moores lov (Som siger at en transistors størrelse
halveres hver 18. måned, og derfor tidligere også forudsagde at en processors
hastighed blev fordoblet hver 18. måned) på sin vis stadig er gyldig, da
computere stadig får mere regnekraft i form af flere kerner. Om dette siger
Vinter dog:\citat{Det lugter af et simpelt regnestykke fra folkeskolen: Hvis én
mand kan grave en grøft på en time, hvor hurtigt kan to mænd så gøre det? En
halv time ville vi nok hurtigt gætte på, men hvad med 1.000 mænd? Kan de gøre
det på 3,6 sekunder? Det er klart for enhver, at man ikke kan grave en grøft på
3,6 sekunder – de mange mennesker ville simpelthen stå i vejen for hinanden.
Sådan er det også, når vi skal parallelisere computerprogrammer; det kan være
meget vanskeligt, selv for de letteste problemer.}{\textit{Supercomputere --
Mange bække små...}, s. 25}

Med andre ord ligger der altså en ny udfordring i forbindelse med at lave
algoritmer og programmer der kan udnytte de mange kerner som computere vil have
i fremtiden. Endvidere kommentere Vinter også på det faktum at hukommelsen i en
computer i dag ikke kan følge med processoren: \citat{Det helt store problem
ligger i selve von Neumann-modellen, nemlig i, at CPU’en hele tiden skal læse
fra hukommelsen -- og desværre er hukommelsen meget langsommere end
CPU’en}{Ibid., s. 26}

Dette løses i dag vha. af en cache-hukommelse, der ligger som lag imellem
processoren og hukommelsen, men endvidere taler Vinter om tre alternativer som
der i øjeblikket kigges på.

Det første er at gøre hukommelsen hurtigere, ved at benytte sig af en lille men
meget hurtig hukommelse til hver processorkerne, og så lade hver af disse have
et lille hjælpekomponent som de bruger til at flytte hukommelsen mellem den
hurtige og den langsomme hukommelse. Dette er en teknik IBM i øjeblikket kigger
på og benyttes bl.a. i Sonys Playstation 3.

Det andet er at gøre hukommelsen hurtigere i gennemsnit (en teknik firmaet
NVidia arbejder på). Dette gør de ved at hente mange flere bytes fra hukommelsen
af gangen (specifikt 448), og derefter deles disse ud til et tilsvarende antal
kerner. Således gøres det op til programmører at skrive programmer der kan
benytte dette.

Det trejde er at gøre processoren relativt langsommere, ved at opdele den i
flere virtuelle kerner som arbejder på hver sin opgave. Dette kan udnyttes af
systemet da processorene gøres langsommere ved at de deles op i virtuelle
processorer. Således kan hukommelsen bedre følge med processoren. Dette betyder
dog igen når man programmere skal skrive det lige så mange kerner som der er
virtuelle kerner.

Det store problem i dissse forskellige teknologier er at de kræver forskellige
måder at programmere på, om hvis man vælger at programmere en til en løsning der
ikke bliver til noget, skal man starte forfra (Et problem der ifølge artiklen
forskes i at løse på DIKU).

\subsection{Kvantecomputeren vs. den fysiske Turingmaskine}
I teksten \textit{Quantum Computing} beskrives fremkomsten og udviklingen 
af den kvantemekaniske computer, samt de typer af problemer den tænkes at
kunne løse.
Kvantecomputeren er grundlæggende set en en teoretisk maskinemodel\footnote{
Introduceret første gang i 1982 af fysikeren Richard Feynman} som benytter 
elementer fra kvantemekanikken (superpositioner, \textit{entanglements}) til 
at repræsentere og operere på data. Groft forenklet adskiller kvantecomputeren
sig fra den klassiske transistorbaserede maskine ved ikke at være klart
deterministisk: Hvor den traditionelle maskines grundlæggende enhed er det binære 
tal (hvis eneste mulige tilstande er 1 og 0) opererer kvantecomputeren på qubits
- en enheden som kan repræsentere et 1, et nul eller en kvantesuperposition af 
begge tilstande. Dette har bl.a. signifikante implikationer for lagring, da én qubit 
kan lagre sit data i flere tilstande simultant: 
\citat{[...] the amount of information that can be stored in a system of  
\textit{n} unmeasured qubits grows exponentially in \textit{n}}{\textit{Quantum Computing} s. 9}
Som teksten 
pointerer, bl.a. i udlægningen af de forskellige konkrete kvantealgoritmer, 
kommer denne forsøgning i mulig lagring ikke uden udgifter - her hovedsageligt 
relateret til hvordan qubitdataet hentes effektivt fra lageret. 

\subsection{Udviklingsperspektiver og begrænsninger}
Det mest essentielle perspektiv i kvantecomputerens udvikling må siges at være
dens potentielle omvæltning af de etablerede grænser for beregnelighed af
umedgørlige problemer.  En kvantecomputer kan, teoretisk set, beregne uhyre
beregningskrævende operationer med eksponentielt speedup for visse kendte
algoritmer \footnote{Oplagte eksempler på sådanne er Shors
faktoriseringsalgoritme og Grovers kvantesøgningsalgoritme.}. Det er dog på
nuværende tidspunkt stadigvæk et åbent spørgsmål hvorvidt kvantecomputeren kan
siges at være mere effektiv end den klassiske Turingmaskine i alle tilfælde, og
ikke blot i isolerede tilfælde med bestemte algoritmer. 

Af begrænsninger ifbm. udviklingen af kvantecomputere kan én fremhæves som særligt
central, navnligt det faktum at kvantecomputeren er meget kompleks at realisere 
som en fysisk maskine: skønt de grundlæggende elementer i en sådan maskine er 
fremstillet, er det pt. ikke lykkedes forskningen at kombinere disse til en samlet
universel maskine\footnote{Se Ibid. s. 19}.

Derudover er en potentiel hindring ved kvantecomputerens dens drastisk mere 
teoretisk involverede fundament ift. den klassiske maskine:  
Som fænomen er kvantecomputeren ganske simpelt betragteligt sværere at forstå, 
og dermed at programmere til.
Som Hagar pointerer: 
\citat{One of the embarrassments of quantum computing is the fact that, so far, 
only one algorithm has been discovered [...] that is significantly faster than 
any \textit{known} classical one. It is almost certain that one of the reasons for
this scarcity of quantum algorithms is related to the lack of our understanding
of what makes a quantum computer quantum.}{Ibid. s. 20}


\section{DNA-computeren}
Den molekulære DNA-computer går kort sagt ud på at bruge DNA-strenge som 
datamodeller/objekter og bruge deres biologiske egenskaber til 
dataprocesser/beregninger.
Det banebrydende eksperiment gik ud på at beregne et lille eksempel på 
Hamilton-sti-problemet ved hjælp af DNA-strenge og biologiske metoder og 
processer; bl.a. ``Watson-Crick pairing'' og ``polymerases'' (alle metoder er
med engelsk navn). Han fulgte algoritmen, hvor man fremstiller en stor mængde
tilfældige veje mellem de mulige knuder, frasortere dem der er for lange og for
korte, ikke starter eller slutter i de givne start- og slutknuder, gennemløber
samme knude flere gange og dem der ikke gennemløber hver knude. Til sidst, hvis
der er nogen veje tilbage i mængden, er svaret at der findes en sti fra start-
til slutknuden der gennemløber alle knuder præcist én gang.
Hver sti mellem to knuder og hver knude(i artiklen er stier beskrevet som flyruter og knuder 
som byer) blev beskrevet ved en bestemt kodning af en DNA-streng. 
Strengen kan symbolsk deles på midten, så knuden har et fornavn, hvor en sti kan
gå til, og et efternavn, hvor en sti kan gå fra. Ligeledes for en sti, beskriver
den første ende af strengen knuden, som stien går fra, og den anden ende
beskriver knuden, som stien går til. Hver DNA-streng har en komplementær
DNA-streng, dens ``Watson-Crick complement'', hvor f.eks. strengen TCGG er
komplimentær med strengen AGCC. For hver knude blev den komplimentære streng
fundet, og sammen med strengene for stierne blev disse produceret (omkring
$10^{14}$ molekyler pr. streng) og ført til et glas vand, med et par yderligere
få ingredienser som salt. Inden for ca. et sekund var løsningen ``beregnet'' i
glasset, idet alle stier fandt sammen med deres ene halvdels komplimentære
streng, der var halvdelen af en knudes streng, og deres anden halvdels
komplimentære, halve knudes streng. Derved blev alle knuder forbundet på de
måder, der var muligt mellem dem, hvilket gav en stor mængde tilfældige veje.
Herpå blev der, ud fra algoritmen og over en uge, sorteret i bl.a. længder og
start- og slutknuder, f.eks. ved at tilføre mikroskopiske jernkugler til de
muligt korrekte DNA-strenge og bruge magnetisme til at trække dem fra resten af
mængden. Til sidst var svaret fundet, hvis der var nogen DNA-strenge tilbage. Og
det var der.


De stærke sider ved DNA-computeren er dels den utroligt hurtige udregning pga. 
sin stærkt paralleliserende metode, og dels pga. den store datalagring på meget 
begrænset plads. Her nævnes i artiklen at et gram DNA, svarende til en 
kubikcentimeter i tørret tilstand, kan lagre data svarende til omkring 1
billion cd'er (hvis en cd svare til ca. 700mb, svare det altså til 700 billioner
mb data eller 667572021 tb data) hvilket er utroligt sammenlignet med dagens
teknologiske muligheder - og prisklasser. Derudover giver følgende citat en god 
idé om potentialet i molekulær-computerens beregninshastighed ift. energien, 
som også især er i fokus for tiden (med energibesparende computere og servere 
og desuden at begrænse varmeudledningen):
\citat{Molecular computers also have the potential for extraordinary energy 
efficiency. In principle, one joule is sufficient for approximately $2*10^{19}$ 
ligation operations. This is remarkable considering that the second law of 
thermodynamics dictates a theoretical maximum of $34*10^{19}$ (irreversible) 
operations per joule (at room temperature). Existing supercomputers are far
less efficient, executing at most $10^{9}$ operations per
joule.}{\textit{Computing with DNA}, s. 61}
Altså har den molekulære computere et meget større potentiale for at opnå 
maksimal energieffektivitet end transistor-computeren.

Man kan forestille sig, at molekulære computere, pga. deres umiddelbart 
langsomme udprintstid (tiden det tager for at modtage resultatet) kan bruges, i 
hvert fald til at begynde med, som en ny slags super-computer. Den kan 
potentielt udregne problemer hurtigere og mere effektivt end eksisterende
(transistor) supercomputere, hvor nogle problematikker tager måneder, år eller
årtier at udregne. Så hvis resultatet kan uprintes hurtigere, end det tager for
en supercomputer at udregne problemet, er DNA-computeren det optimale valg.
Spørgsmålet er selvfølgelig hvor godt den skalere ift. størrelsen af problemet,
men umiddelbart kan processerne være lige så simple og ``hurtige'' for
udprintning, som det var for det begrænsede forsøg med Hamilton-sti-problemet.

Derved kan man udnytte dens hurtige regnehastighed, selvom resultatet er
besværligt at udprinte. Derimod vil man i den kommercielle computerverden få
svært ved at bruge DNA-computeren, idet beregningerne kræver markant mindre. Det
går f.eks. ikke, at det tager en uge at få opdateret beløbet på sin bankkonto,
efter man har trukket penge fra den, mens det for f.eks. beregninger til
kortlægning af hjernen eller udregning af logistiske problemer (som
Hamilton-sti-problemet, blot større) i forvejen tager meget lang tids udregning.

\section{Diskussion og konklusion}
Alle fire artikler i nærværende opgave arbejder med et udtalt fokus på
computeren som en regnemaskine og hvordan dennes effektivitet og lagerkapacitet kan 
forøges maksimalt. I teksterne omhandlende den klassiske transistorbaserede
computer tager dette form som et fokus på et beskrive udviklingen til maskiner
med flere kerner, og hvordan besværet med at udnytte hukommelse i denne forbindelse
kan bearbejdes. Et andet aspekt i skiftet til talrige kerner er hvordan man bør
skrive nye algoritmer der optimalt arbejder med de distribuerede resurser. 
Dette fokus på forøgning af hastighed og beregningskraft er også til stede i
teksten omhandlende kvantecomputere - her understreges dog også spørgsmålet om 
hvorvidt en kvantecomputer vil være i stand til at regne typer af problemer som 
den transistorbaserede (selv med adskillige kerner) ikke formår at regne i 
polynomiel tid. Lig teksten om klassiske computere fremhæver teksten om 
kvantecomputere nødvendigheden af at skrive nye algoritmer, som arbejder 
specifikt for at udnytte mulighederne i den kvantemekaniske computermodel. 
Også i beskrivelsen af DNA-baseret beregning er de biomolekylære strukturers
mulighed for at øge speedup af regnekraft i centrum, dog med den essentielle
hindring at fremstillingen af udprint fra beregningerne er langsomme og 
omkostningsrige. 
\end{document}
