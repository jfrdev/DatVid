\documentclass[10pt,a4paper]{article}
\usepackage[danish]{babel}
\usepackage{ucs}
\usepackage[utf8x]{inputenc}
\usepackage[T1]{fontenc}
\usepackage[danish]{babel}
\usepackage{libertine}
\usepackage[parfill]{parskip}
\usepackage{ragged2e}
\widowpenalty=1000
\clubpenalty=1000

\newcommand{\citat}[2]{\begin{justify}\textit{``#1''}\hspace{0.1cm}\footnote{#2}\end{justify}}

\title{Gruppeopgave 5}
\author{Naja Wulff Mottelson (300988)
        Jens Fredskov () 
        Henrik Bendt ()}
\date{\today}
\setlength\parskip{0em}
\setlength\parindent{0em}


\begin{document}
\maketitle

\section{Spørgsmål 1}
Vi vurderer at persondataloven fungerer med tilbagevirkende kraft, 
selv tilbage til 1955. FCM må derfor som udgangspunkt ikke 
offentliggøre billeder (da disse tæller som personoplysninger) uden
samtykke fra de portrætterede\footnote{Jvf. Persondatalovspjece s. 13}.

\paragraph{}
Et mindre aber dabei at anføre her er at vi ikke (ud fra opgaveformuleringen)
ved hvorvidt FCM er 
registreret som en forening eller et privat firma. Eftersom specielle 
regler gælder for foreninger kan dette potentielt påvirke hvad 
de har tilladelse til. Her og i det følgende antager vi at FCM
er et privat firma. 

\section{Spørgsmål 2}
Såfremt billedet er offentligt tilgængeligt, og de oplysninger 
Egon Hansen lægger op kan udtrækkes fra offentligt tilgængeligt
materiale udgivet med samtykke fra Brian Hansen (f.eks. interviews, 
dagbladsportrætter o. lign.) er det lovligt at lægge op på
FCMs hjemmeside. Ellers skal samtykke indhentes forud. 

\section{Spørgsmål 3}
Vi vurderer at oplysninger om skader samt antal af modtagne gule 
kort er følsomme persondata, men begrundet som relevant data for
FCM som firma. Opbevarelsen af spillernes personnumre forudsætter
at personnumrene er indhentet med eksplicit samtykke fra spillerne. 

\section{Spørgsmål 4}
FCM må som udgangspunkt udelukkende behandle almindelige oplysninger 
om fans hvis disse er 
defineret som klubbens kunder eller hvis pågældende fans har givet
deres eksplicitte samtykke\footnote{Se Persondatalovspjece s. 14}.
 Eftersom fans betaler FCM for at komme til 
kampe, samt gerne betaler et kontingent, vil de umiddelbart godt 
kunne defineres som kunder i denne forbindelse. 

\paragraph{}
I fan-registret lagres ligeledes CPR-numre, hvilket giver anledning
til samme problematik som i spillerrgistret. Opbevaringen af information
om hvorvidt en fan har lavet ballade vurderer vi er relevant information
for FCM, da dette kan skulle kunne bruges til at møde et retskrav
f.eks. ifbm. karantæne. I denne forbindelse mener vi at klubbens interesse
overstiger det enkelte klubmedlems (jvf. slide nr. 13 fra denne uges 
forelæsning). 

\section{Spørgsmål 5}
Vi vurderer at sponsorregisteret er fuldt lovligt, såfremt de forskellige 
sponsorer har givet tilladelse til indhentningen af deres CVR-nummer
(igen med begrundelse i samme overvejelser som ifbm. spiller- og fan-registret). 

\section{Spørgsmål 6}
Såfremt Johnny er af den opfattelse at informationen om øldåsen er forkert, 
(og er i stand til at dokumentere det) vil han kunne bede FCM om at fjerne
oplysningen fra registret. Han kan dog blive nødsaget til at gennemgå en
retssag for at fastslå legitimiteten af sin forespørgsel. 

\section{Spørgsmål 7}
Eftersom privatpersoner altid har ret til indsigt i de oplysninger der behandles 
om dem, vil Johnny have ret til at få at vide hvorvidt han har karantæne\footnote{
Se Persondatalovspjece s. 24}.

\section{Spørgsmål 8}
Iflg. opgaveformuleringen  har både sponsorer og Målby Sparekasse har adgang til 
f.eks. fan-registeret. Eftersom følsom data opbevares her, er sikkerheden derfor
ikke tilstrækkelig medmindre de registrerede personer har givet eksplicit 
tilladelse til at oplysningerne om dem videregives.
\end{document}
